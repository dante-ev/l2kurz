% master: l2kurz.tex
% L2K4.TEX - 4.Teil der LaTeX2e-Kurzbeschreibung v2.2
% L2K4.TEX - 4.Teil der LaTeX2e-Kurzbeschreibung Mainz 1994, 1995
% LK4.TEX  - 4.Teil der LaTeX-Kurzbeschreibung Graz-Wien 1987
% last changes: 2001-06-10 (WaS)

\section{Setzen von Bildern} \label{graphics}
Lädt man im Vorspann des Dokuments das Paket \texttt{graphicx},
dann kann man Bilder, die mit einem anderen Programm erzeugt wurden, 
mit dem Befehl \verb|\includegraphics| in das Dokument einfügen.
Der Dateiname ist als Argument des Befehls anzugeben.
Welche Dateiformate verarbeitet werden können, hängt vom jeweiligen
\TeX-System ab und muss in dessen Dokumentation beschrieben sein\todo{PG: wir können m.E. schon angeben: bei PDFTeX kann das PDF, PNG und JPEG sein}.

\begin{LTXexample}
Hier \includegraphics{a} ist ein Bild.
\end{LTXexample}

\todo[inline]{PG: hier ein schöneres Bild einbinden}

\noindent Wird das Paket \texttt{graphicx} mit der Option \texttt{[draft]} geladen,
dann erscheint anstelle des Bildes nur ein Rahmen entsprechend
der tatsächlichen Bildgröße mit dem Namen des Grafikfiles, 
was die Bearbeitung beschleunigt und für Probeausdrucke nützlich ist.

Weitere Informationen zum Einbinden von Bildern finden Sie in der
Online"=Dokumentation \cite{grfguide}, im \textit{Graphics Companion}
\cite{grfcomp} und in K.~Reckdahls empfehlenswertem  Tutorium \cite{epslatex}.



