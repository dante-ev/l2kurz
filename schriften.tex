%!TEX root = l2kurz.tex

% master: l2kurz.tex
% l2k5.tex - 5.Teil der LaTeX2e-Kurzbeschreibung v2.3
% 2001-04-10 (WaS)

\section{Schriften}
Normalerweise wählt \LaTeX\ die Größe und den Stil der Schrift
aufgrund der Befehle aus, die die logische Struktur des Textes angeben:
Überschriften, Fußnoten, Hervorhebungen usw.
Im folgenden werden Befehle und Makropakete beschrieben, mit denen
die Schrift auch explizit beeinflusst werden kann.
Ausführlichere Erläuterungen zum Umgang mit Schriften in \LaTeX{}
findet man im \textit{\LaTeX-Begleiter} \cite{wonne}
und in der Online-Dokumentation \cite{fntguide}. Für \XeLaTeX{} und \LuaLaTeX{} steht das Paket
\texttt{fontspec} zur Verfügung, mit dem beliebige OpenType und TrueType Schriftarten eingebunden werden können, ohne dass sie für \LaTeX{} angepasst werden müssen. 

\begin{lstlisting}
\documentclass{article}
\usepackage[ngerman]{babel}
% Nur XeLaTeX und LuaLaTeX:
\usepackage{fontspec}
\setmainfont{Linux Libertine O}
\begin{document}
Das ist ein Text in der Schriftart Linux Libertine mit den Schnitten
\textit{Kursiv}, \textbf{\textit{Fett Kursiv}} und \textsc{Kapitälchen}.
\end{document}
\end{lstlisting}

Ergibt einen Text in der Schriftart Linux Libertine:

\medskip

\noindent\includegraphics{fontspec}

\subsection{Schriftgrößen}
 
Die in der Tabelle~\ref{sizes} angeführten Befehlen 
wechseln die Schriftgröße.
Sie spezifizieren die Größe relativ
zu der von \lstinline:\documentclass: festgelegten Grundschrift.
Ihr Wirkung reicht bis zum Ende der aktuellen Gruppe oder Umgebung.


\begin{table}[!htb]
\caption{Schriftgrößen} \label{sizes}
\def\arraystretch{1.25}
\centering
\begin{tabular}{@{}ll@{}}
\toprule
\lstinline|\tiny|         & \tiny        winzig kleine Schrift \\
\lstinline|\scriptsize|   & \scriptsize  sehr kleine Schrift (wie Indizes)\\
\lstinline|\footnotesize| & \footnotesize     kleine Schrift (wie Fußnoten)\\
\lstinline|\small|        & \small            kleine Schrift \\
\lstinline|\normalsize|   & \normalsize  normale Schrift \\
\lstinline|\large|        & \large       große Schrift \\
\lstinline|\Large|        & \Large       größere Schrift \\
\lstinline|\LARGE|        & \LARGE       sehr große Schrift \\[3pt]
\lstinline|\huge|         & \huge        riesig groß \\[3pt]
\lstinline|\Huge|        & \Huge        gigantisch \\
\bottomrule
\end{tabular}
\end{table}
 
Die Größen-Befehle verändern auch die Zeilenabstände auf
die jeweils passenden Werte -- aber nur, wenn die
Leerzeile, die den Absatz beendet, innerhalb des
Gültigkeitsbereichs des Größen-Befehls liegt:

\begin{LTXexample}
{\Large zu enger \\
Abstand}\par
\end{LTXexample}

\begin{LTXexample}
{\Large richtiger\\
Abstand\par}
\end{LTXexample}

Für korrekte Zeilenabstände darf die
schließende geschwungene Klammer also nicht zu früh kommen,
sondern erst nach einem Absatzende, das übrigens nicht nur als
Leerzeile, sondern auch als Befehl \lstinline|\par|  eingegeben werden 
kann.


\subsection{Schriftstil}
Der Schriftstil wird in \LaTeX{} durch 3~Merkmale definiert:
\begin{description}
\item[Familie] Standardmäßig stehen 3~Familien zur Wahl:
  "`roman"' (Antiqua), "`sans serif"' (Serifenlose) und "`typewriter"'
  (Schreibmaschinenschrift).
\item[Serie] Die Serie gibt Stärke und Laufweite der
  Schrift an: "`medium"' (normale Schrift), "`boldface extended"'
  (fett und breiter).
\item[Form] Die Form der Buchstaben: "`upright"'
  (aufrecht), "`slanted"' (geneigt), "`italic"' (kursiv),
  "`caps and small caps"' (Kapitälchen).
\end{description}
Tabelle~\ref{fonts} zeigt die Befehle, mit denen diese Attribute 
explizit beeinflusst werden können.  
Die Befehle der Form \lstinline|\text...| setzen nur ihr Argument im 
gewünschten  Stil.  Zu jedem dieser Befehle ist ein Gegenstück angegeben, 
das von seinem Auf\/treten an bis zum Ende der laufenden Gruppe oder Umgebung 
wirkt.

Zu beachten ist, dass Wörter in Schreibmaschinenschrift nicht automatisch
getrennt werden.\par

\begin{table}[hbp]
\caption{Schriftstile} \label{fonts}
\def\arraystretch{1.25}
\centering
\begin{tabular}{@{}lll@{}}
\toprule
\lstinline|\textrm|\{\textit{text}\}     &\lstinline|\rmfamily|   &\textrm{Antiqua}\\
\lstinline|\textsf|\{\textit{text}\}     &\lstinline|\sffamily|   &\textsf{Serifenlose}\\
\lstinline|\texttt|\{\textit{text}\}     &\lstinline|\ttfamily|   &\texttt{Maschinenschrift}\\
\lstinline|\textmd|\{\textit{text}\}     &\lstinline|\mdseries|   &\textmd{normal}\\
\lstinline|\textbf|\{\textit{text}\}     &\lstinline|\bfseries|   &\textbf{fett, breiter laufend}\\
\lstinline|\textup|\{\textit{text}\}     &\lstinline|\upshape|    &\textup{aufrecht}\\
\lstinline|\textsl|\{\textit{text}\}     &\lstinline|\slshape|    &\textsl{geneigt}\\
\lstinline|\textit|\{\textit{text}\}     &\lstinline|\itshape|    &\textit{kursiv}\\
\lstinline|\textsc|\{\textit{text}\}     &\lstinline|\scshape|    &\textsc{Kapitälchen}\\
\lstinline|\textnormal|\{\textit{text}\} &\lstinline|\normalfont| &\textnormal{Die Grundschrift des Dokuments}\\
\bottomrule
\end{tabular}
\end{table}

Die Befehle für Familie, Serie und Form können untereinander und mit den
Größen-Befehlen kombiniert werden;  allerdings muss nicht jede
mögliche Kombination tatsächlich als reale Schrift (Font)
zur Verfügung stehen.

\begin{LTXexample}
{\small Die kleinen \textbf{fetten}
Römer
beherrschten }{\large das
ganze gro"se \textit{Italien}.}
{\Large\sffamily\slshape plakativ}
\end{LTXexample}


Je \emph{weniger} verschiedene Schriftarten man verwendet, desto
lesbarer und schöner wird das Schriftstück!


\subsection{Andere Schriftfamilien}
Mit den im vorigen Abschnitt eingeführten Befehlen kann man nicht beeinflussen,
welche Schriftfamilien tatsächlich als Antiqua, Serifenlose und
Maschinenschrift benutzt werden.  \LaTeX{} verwendet als Voreinstellung
die sog.\ Computer-Modern-Schriftfamilien (CM), siehe Tabelle~\ref{families};
der Stil der mathematischen Zeichensätze passt dabei zu CM~Roman.

Will man andere Schriften benutzen, dann ist der einfachste Weg 
das Laden eines Pakets, das eine oder mehrere dieser Schriftfamilien 
komplett ersetzt.
Tabelle~\ref{families} führt einige derartige Pakete auf%,
% die allerdings nicht in jeder \LaTeX-Installation verfügbar sein müssen
.

Die Dokumentation der \TeX"=Distributionen sollte darüber
informieren, welche Schriften verfügbar sind
und wie Sie weitere installieren und verwenden können.
Insbesondere sollte eine Anzahl von verbreiteten PostScript-Schriften
mit jedem aktuellen \LaTeX-System verwendbar sein \cite{postscript}.


\begin{table}[htb]
\caption[Pakete für alternative Schriftfamilien]
{Pakete für alternative Schriftfamilien \newline\small(Eine leere
Tabellenspalte bedeutet, dass das Paket die betreffende Schriftfamilie nicht 
verändert; * kennzeichnet die jeweils als Grundschrift eingestellte Familie.)}
\label{families}
{\footnotesize
\medskip
\renewcommand{\arraystretch}{1.5}
\newcolumntype{x}{>{\RaggedRight\hspace{0pt}}X}
\begin{tabularx}{\textwidth}{@{}lxxxx@{}}
\toprule
Paket            & Antiqua    & Serifenlose   & Schreibmaschine  & math.\ Formeln\tabularnewline\midrule
(keines)         & CM Roman * & CM Sans Serif & CM Typewriter    & $\approx$ CM Roman\tabularnewline
% \texttt{ccfonts} & Concrete *
%                  &
%                  &
%                  & $\approx$ Concrete\\
% \texttt{cmbright}&
%                  & CM Bright *
%                  & {\raggedright CM\ Typewriter\\ Light}
%                  & $\approx$ CM Bright\\
\texttt{courier} &
                 &
                 & Courier 
                 & \tabularnewline
\texttt{droid}   & Droid Serif~*  & Droid Sans    & Droid Sans Mono &                  \tabularnewline
\texttt{fourier} & Utopia Regular~*    &               &                 & Fourier          \tabularnewline
\texttt{helvet}  & 
                 & Helvetica
                 &
                 & \tabularnewline
\texttt{inconsolata} &                 &                & Inconsolata    &                  \tabularnewline
\texttt{libertine} & Linux Libertine~* & Linux Biolinum &                &                  \tabularnewline
\texttt{lmodern} & LM Roman *          & LM Sans Serif & LM Typewriter & $\approx$ LM Roman \tabularnewline
\texttt{mathptmx}& Times *
                 &
                 &
                 & $\approx$ Times\tabularnewline
\texttt{mathpazo}& Palatino *
                 &
                 &
                 & $\approx$ Palatino\tabularnewline
\bottomrule                
\end{tabularx}
}
\end{table}


\subsection{Die "`europäischen"' Zeichensätze}
\LaTeX{} verwendet standardmäßig  Schriften mit einem Umfang von
128~Zeichen.  Umlaute oder akzentuierte Buchstaben sind darin nicht
enthalten; sie werden jeweils aus dem Grundsymbol und dem Akzent
zusammengesetzt.  

Inzwischen stehen die meisten der mit \LaTeX\ verwendbaren Schriften
auch mit einem erweiterten "`europäischen"' Zeichenvorrat bereit.
Sie enthalten jetzt 256 Zeichen, welche fast
alle europäischen Sprachen abdecken, d.\,h., jedes benötigte
Zeichen ist vorgefertigt in ihnen enthalten.
Das hat nicht nur eine
höhere typographische Qualität zur Folge; aufgrund der inneren Arbeitsweise
von \TeX{} entfallen damit auch die Einschränkungen im Zusammenhang mit
der Silbentrennung, die im Abschnitt~\ref{silb} erwähnt wurden:
Wörter mit Umlauten werden nun besser getrennt, und im Argument des
Befehls \lstinline|\hyphenation| dürfen auch Umlaute und das scharfe~s stehen.

%% stimmt nur fuer EC
%Weiterhin sind die Unterschneidungen im Vergleich zu den amerikanischen
%\TeX-Originalschriften stark verbessert und nun auch auf häufige
%Buchstabenpaarungen in nicht-englischen Sprachen optimiert.

% <------- Formulierung ????
Die europäischen Schriften bestehen aus zwei Teilen: Der T1-Zeichensatz
enthält Buchstaben, ASCII-Zeichen sowie verschiedene Anführungszeichen
und Striche, 
während ein ergänzender TS1-Zeichensatz zusätzliche Textsymbole bereitstellt.
% <------- Formulierung ????

\LaTeX{} wird veranlasst, T1-Schriften zu verwenden,
indem man das Paket \texttt{fontenc} mit der Option \texttt{T1} lädt:
\begin{quote}
  \lstinline|\usepackage[T1]{fontenc}|
\end{quote}
Das Paket \texttt{textcomp} ermöglicht den Zugriff auf die Textsymbole:
\begin{quote}
  \lstinline|\usepackage{textcomp}|
\end{quote}
Welche zusätzlichen Zeichen mit den T1-Schriften
bereitgestellt werden, ist in \cite{usrguide} zusammengefasst;
Anhang~\ref{textsymbols} der vorliegenden Kurzbeschreibung
enthält eine Liste aller TS1-Textsymbole.  Einige der Textsymbole sind
auch ohne das Paket \texttt{textcomp} verfügbar, siehe Abschnitt~\ref{symbole},
dann aber nicht immer in einem zur laufenden Schrift passenden Stil.

Beachten Sie, dass in Fonts, die nicht speziell für die Verwendung 
mit \TeX\ entworfen wurden, 
nur ein Teil der TS1-Textsymbole enthalten ist.
Das betrifft vor allem die "`handelsüblichen"' PostScript-Schriften.

\endinput

